%This work is licensed under the Creative Commons
%Attribution-ShareAlike 4.0 International License. To view a copy of
%this license, visit http://creativecommons.org/licenses/by-sa/4.0/ or
%send a letter to Creative Commons, PO Box 1866, Mountain View, CA
%94042, USA.

%This work is licensed under the Creative Commons
%Attribution-ShareAlike 4.0 International License. To view a copy of
%this license, visit http://creativecommons.org/licenses/by-sa/4.0/ or
%send a letter to Creative Commons, PO Box 1866, Mountain View, CA
%94042, USA.

%\documentclass[gray,handout, pdftex, 11pt]{beamer}
%\documentclass[handout, pdftex, 11pt]{beamer}

\documentclass[pdflatex, 11pt]{beamer}

\usepackage[utf8]{inputenc}
\usepackage[T1]{fontenc}
\usepackage{calligra}
\usepackage{lmodern}
\usepackage[italian]{babel}
\usepackage{acronym}
\usepackage{graphicx}
\usepackage{multirow}
\usepackage{listings}
\usepackage{microtype}
\usepackage{acronym}
\usepackage{array}
\usepackage{amsmath}
\usepackage{appendixnumberbeamer}
%\usepackage{latexsym}
\usepackage{tikz}
\usetikzlibrary{shapes, chains, scopes, shadows, positioning, arrows,
  decorations.pathmorphing, calc, mindmap, petri}

\newcommand{\bigO}{\ensuremath{\mathcal{O}}}
\newcommand{\ctlNext}{\ensuremath{\bigcirc}}
\newcommand{\ctlUntil}{\ensuremath{\,\mathbf{U}\,}}
\newcommand{\ctlWeakUntil}{\ensuremath{\,\mathbf{W}\,}}
\newcommand{\ctlAlways}{\ensuremath{\Box}}
\newcommand{\ctlEventually}{\ensuremath{\Diamond}}
\DeclareMathOperator*{\landatory}{\hbox{\huge $\land$}}
\DeclareMathOperator*{\lorratory}{\hbox{\huge $\lor$}}

\def\transW{8mm}
\def\transH{2mm}

\colorlet{mmcb}{black!70}
\colorlet{mmc1}{red!80}
\colorlet{mmc2}{blue!80}
\colorlet{c1}{green!20}
\colorlet{c2}{blue!10}
\colorlet{c3}{yellow!10}
\colorlet{c4}{red!10}
\colorlet{drawColor}{black!80}
\colorlet{commentColor}{green!70!black!90}
\colorlet{codeBgColor}{yellow!20}
\colorlet{bashBgColor}{green!50}
\colorlet{mcI}{yellow!60}
\colorlet{mcOok}{green!60}
\colorlet{mcOko}{red!60}
\colorlet{mcB}{blue!10}
\colorlet{mcB2}{blue}

\tikzset{onslide/.code args={<#1>#2}{%
  \only<#1>{\pgfkeysalso{#2}} % \pgfkeysalso doesn't change the path
}}
\tikzset{temporal/.code args={<#1>#2#3#4}{%
  \temporal<#1>{\pgfkeysalso{#2}}{\pgfkeysalso{#3}}{\pgfkeysalso{#4}} % \pgfkeysalso doesn't change the path
}}

\tikzstyle{alertStar}=[circle, decorate, decoration={zigzag,segment length=3.12mm,amplitude=1mm}, align=center, drop shadow, draw=drawColor, fill=white]
\tikzstyle{oval}=[ellipse, align=center, drop shadow, draw=drawColor, fill=white]
\tikzstyle{rect}=[rectangle, rounded corners=2pt, align=center, drop
shadow, draw=drawColor, fill=white]
\tikzstyle{arrow}=[->, very thick, >=stealth', draw=black!80]
\tikzstyle{bigArrow}=[->, very thick, >=stealth', draw=red]
\tikzstyle{myMindmap}=[mindmap,
every node/.style={concept, minimum size=5mm, text width=5mm}, 
% every child/.style={level distance=10mm, concept color=mmcb}
level 1/.append style={level distance=10mm,sibling angle=45},
level 2/.append style={level distance=10mm,sibling angle=45},
level 3/.append style={level distance=10mm,sibling angle=45}
]
\tikzstyle{myPlace} = [place, very thick, draw=drawColor, fill=white, drop shadow]
\tikzstyle{transExpH} = [transition, very thick, draw=drawColor, fill=white, drop
shadow, minimum width=\transW, minimum height=\transH]
\tikzstyle{transExpV} = [transition, very thick, draw=drawColor, fill=white, drop
shadow, minimum width=\transH, minimum height=\transW]
\tikzstyle{transDetH} = [transition, very thick, draw=drawColor, fill=black, drop shadow, minimum width=\transW, minimum height=\transH]
\tikzstyle{transDetV} = [transition, very thick, draw=drawColor, fill=black, drop shadow, minimum width=\transH, minimum height=\transW]
\tikzstyle{pre}=[<-, very thick, >=stealth', draw=drawColor]
\tikzstyle{preN}=[<-, very thick, >=o, draw=drawColor]
\tikzstyle{post}=[->, very thick, >=stealth', draw=drawColor]
\tikzstyle{highlight}=[draw=red]

\lstdefinestyle{customConf}{
   basicstyle=\scriptsize\ttfamily,
   breaklines=true, breakatwhitespace=true,
   postbreak=\raisebox{0ex}[0ex][0ex]{\ensuremath{\color{red}\hookrightarrow\space}},
   frameround=fttt,
   frame=trBL,
   backgroundcolor=\color{yellow!20},
   numbers=left,
   stepnumber=1,    
   firstnumber=1,
   numberfirstline=true,
   numberstyle=\tiny\color{black!50},
   xleftmargin=1.75em,
   framexleftmargin=2.1em,
   % rulesepcolor=\color{gray},
   rulecolor=\color{black}
   % linewidth=8cm,
}

\lstdefinestyle{customPython}{
   language=Python,
   % basicstyle=\small\ttfamily\bfseries,
   basicstyle=\tiny\ttfamily,
   keywordstyle=\color{blue}\ttfamily,
   stringstyle=\color{red}\ttfamily,
   commentstyle=\color{green}\ttfamily,
   morecomment=[l][\color{magenta}]{\#},
   % breaklines=false,
   breaklines=true, breakatwhitespace=true,
   postbreak=\raisebox{0ex}[0ex][0ex]{\ensuremath{\color{red}\hookrightarrow\space}},
   frameround=fttt,
   frame=trBL,
   backgroundcolor=\color{yellow!20},
   numbers=left,
   stepnumber=1,    
   firstnumber=1,
   numberfirstline=true,
   numberstyle=\tiny\color{black!50},
   xleftmargin=2.75em,
   framexleftmargin=3.1em,
   % rulesepcolor=\color{gray},
   rulecolor=\color{black}
   % linewidth=8cm,
}

\lstdefinestyle{customInlinePython}{
   language=Python,
   % basicstyle=\small\ttfamily\bfseries,
   basicstyle=\ttfamily,
   keywordstyle=\color{blue}\ttfamily,
   stringstyle=\color{red}\ttfamily,
   commentstyle=\color{green}\ttfamily,
   morecomment=[l][\color{magenta}]{\#}
}

\lstnewenvironment{pblock}[1][]
{
  \lstset{
    style=customPython,
    #1
  }
}{}

\newcommand{\pfile}[2][]{
  \lstinputlisting[style=customPython, title={\texttt{\detokenize{#2}}}, #1]{#2}
}

\newcommand{\pfilePart}[3]{
  \lstinputlisting[style=customPython, firstnumber=#1, firstline=#1,lastline=#2]{#3}
}

\newcommand{\pp}[2][]{%\lstinline[style=customInlinePython,#1]`#2`}
  \colorbox{codeBgColor}{
    \lstinline[style=customInlinePython,#1]`#2`
  }
}

\graphicspath{{img/}}
%\lstset{inputpath=../cSrc/}

\definecolor{links}{HTML}{2A1B81}
\hypersetup{colorlinks,linkcolor=links,urlcolor=links}

\definecolor{links}{HTML}{2A1B81}
\hypersetup{colorlinks,linkcolor=,urlcolor=links}


\mode<presentation>{
  \usetheme{Martina}
  \setbeamercovered{transparent}
  %\setbeameroption{show notes on second screen=right}
  %\setbeamertemplate{navigation symbols}{}
  %\setbeamertemplate{footline}{}

  \bibliographystyle{abbrv}  
  %\renewcommand\bibfont{\scriptsize}
  \setbeamertemplate{bibliography item}{\textbullet}
  \setbeamertemplate{itemize item}{\checkmark}
  % \setbeamertemplate{itemize subitem}{-}
  \setbeamertemplate{enumerate items}[default]
  \setbeamertemplate{sections/subsections in toc}[square]
}



\begin{document}

\title[CTL model checking]{\textbf{CTL Model Checking}}
\date[\today]{\flushright \today}
%\subtitle{}
\institute[Uni. Firenze]{
  \includegraphics[width=5cm]{img/logoUnifiName.eps}
}

\author[Buracchi M. - Martina S.]{
  \begin{center}
    \begin{tabular}{lr}
      Marco \textsc{Buracchi} & Stefano \textsc{Martina}\\
      \href{mailto:marco.buracchi1@stud.unifi.it}{marco.buracchi1@stud.unifi.it}&
      \href{mailto:stefano.martina@stud.unifi.it}{stefano.martina@stud.unifi.it}\\
    \end{tabular}
  \end{center}
}

% \titlegraphic{
%   \vspace{-0.5cm}
%   \tiny
%   \href{http://creativecommons.org/licenses/by-sa/4.0/}{\includegraphics[width=1cm]{img/logoCC.png}}
%   This work is licensed under a
%   \href{http://creativecommons.org/licenses/by-sa/4.0/}{Creative
%     Commons Attribution-ShareAlike 4.0 International License}.
% }

\newacro{CTL}{Computation Tree Logic}
\newacro{ENF}{Existential Normal Form}
\newacro{TS}{Transition System}
\newacro{LTS}{Labelled Transition System}
\newacro{LTL}{Linear Temporal Logic}
\newacro{HP}{Problema Hamiltoniano}

%\acrodefplural{VD}[VDs]{Voronoi Diagrams}

\begin{frame}[plain]
  \titlepage
\end{frame}

\section{Introduzione}
\begin{frame}
  \frametitle{Model checking}
    \begin{tikzpicture}
      \node(req) at (-1,5) [oval, fill=mcI] {Requirements};
      \node(form) at (-1,3.5) [rect] {Formalizing};
      \node(prop) at (-1,2) [oval,label={[label distance=-1mm]30:\color{mcB2}\emph{formule \acs{CTL}}}] {Property\\specification};
      \node(mc) at (2,1) [rect, fill=mcB] {Model checking};
      \node(sat) at (-2,-1) [oval, fill=mcOok] {Satisfied};
      \node(vio) at (2,-1) [oval, fill=mcOko] {Violated +\\counterexample};
      \node(sim) at (6,0.5) [rect] {Simulation};
      \node(err) at (8,-1) [oval] {Location\\error};
      \node(sysmod) at (4.5,2) [oval,label={[label distance=-1mm]30:\color{mcB2}\emph{\acl{TS}}}] {System\\model};
      \node(mod) at (4.5,3.5) [rect] {Modeling};
      \node(sys) at (4.5,5) [oval, fill=mcI] {System};

      \draw[arrow] (req) to (form);
      \draw[arrow] (form) to (prop);
      \draw[arrow] (prop) |- (mc);
      \draw[arrow] (mc) to (sat);
      \draw[arrow] (mc) to (vio);
      \draw[arrow] (vio) to (sim);
      \draw[arrow] (sim) to (err);
      \draw[arrow] (sysmod) |- (mc);
      \draw[arrow] (sysmod) -| (sim);
      \draw[arrow] (mod) to (sysmod);
      \draw[arrow] (sys) to (mod);
    \end{tikzpicture}
\end{frame}

\begin{frame}
  \frametitle{\ac{CTL}}
  \begin{block}{Grammatica}
   \begin{eqnarray*}
     \Phi &::=& true|a|\Phi_1\land
     \Phi_2|\neg\Phi|\exists\varphi|\forall\varphi\\
     \varphi &::=& \ctlNext\Phi|\Phi_1\ctlUntil\Phi_2
   \end{eqnarray*}
  \end{block}
  \begin{itemize}
  \item $\Phi$ sono state formula
  \item $\varphi$ sono path formula
  \end{itemize}
  \begin{block}{Operatori derivati}
    \begin{eqnarray*}
      \exists\ctlEventually\Phi &=& \exists(true\ctlUntil\Phi)\\
      \forall\ctlEventually\Phi &=& \forall(true\ctlUntil\Phi)\\
      \exists\ctlAlways\Phi &=& \neg\forall\ctlEventually\neg\Phi\\
      \forall\ctlAlways\Phi &=& \neg\exists\ctlEventually\neg\Phi
    \end{eqnarray*}
  \end{block}
\end{frame}

\begin{frame}
  \frametitle{\ac{CTL} \ac{ENF}}
  \begin{block}{Grammatica}
   \begin{eqnarray*}
     \Phi &::=& true|a|\Phi_1\land
     \Phi_2|\neg\Phi|\exists\ctlNext\Phi|\exists(\Phi_1\ctlUntil\Phi_2)|
     \exists\ctlAlways\Phi
   \end{eqnarray*}
  \end{block}
  \begin{block}{Equivalenze}
    \begin{eqnarray*}
      \forall\ctlNext\Phi&\equiv&\neg\exists\ctlNext\neg\Phi\\
      \forall\ctlEventually\Phi&\equiv&\neg\exists\ctlAlways\neg\Phi\\
      \forall\ctlAlways\Phi&\equiv&\neg\exists\ctlEventually\neg\Phi=\neg\exists(true\ctlUntil\neg\Phi)
    \end{eqnarray*}
  \end{block}
  \begin{itemize}
  \item utile anche la formula vista prima: $\exists\ctlEventually\Phi = \exists(true\ctlUntil\Phi)$
  \end{itemize}
\end{frame}

\begin{frame}
  \frametitle{Equivalenze non banali}
  \begin{block}{$\forall$ until}
    \begin{eqnarray*}
      \forall(\Phi\ctlUntil\Psi)&\equiv&\neg\exists(\neg\Psi\ctlUntil(\neg\Phi\land\neg\Psi))\land\neg\exists\ctlAlways\neg\Psi
    \end{eqnarray*}
  \end{block}
  \begin{block}{Weak until}
    \begin{eqnarray*}
    \exists(\Phi\ctlWeakUntil\Psi) &=&
    \neg\forall((\Phi\land\neg\Psi)\ctlUntil(\neg\Phi\land\neg\Psi))\\
    &=&\exists(\Phi\ctlUntil\Psi)\lor\exists\ctlAlways\Phi\\
    &=&\neg(\neg\exists(\Phi\ctlUntil\Psi)\land\neg\exists\ctlAlways\Phi)\\
    \forall(\Phi\ctlWeakUntil\Psi) &=& \neg\exists((\Phi\land\neg\Psi)\ctlUntil(\neg\Phi\land\neg\Psi))
    \end{eqnarray*}
  \end{block}
  \begin{itemize}
  \item Comportano esplosione \alert{esponenziale} della formula
  \end{itemize}
\end{frame}

\begin{frame}[fragile]
  \frametitle{Formula example}
  \begin{block}{Formula}
    \begin{equation*}
      \Phi=\exists\ctlNext a
      \land\exists(b\ctlUntil\exists\ctlAlways\neg c)\ \equiv\ \text{\pp{EX(a) \& (b EU (EG !c))}}
    \end{equation*}    
  \end{block}
  \begin{columns}
    \begin{column}{0.4\textwidth}
      \vspace{-1cm}
      \begin{center}
        \begin{tikzpicture}[scale=0.8, sibling distance=3cm]
          \node[oval,label={[label distance=-2mm]110:$f_0$}](and) {$\land$} 
          child{ node[oval,label={[label distance=-1mm]110:$f_1$}] {$\exists\ctlNext$}
            child { node[oval,label={[label distance=-2mm]110:$f_3$}] {$a$} }
          }
          child{ node[oval,label={[label distance=-2mm]80:$f_2$}] {$\exists\ctlUntil$}
            child { node[oval,label={[label distance=-2mm]110:$f_4$}] {$b$} }
            child { node[oval,label={[label distance=-2mm]80:$f_5$}] {$\exists\ctlAlways$}
              child { node[oval,label={[label distance=-2mm]80:$f_6$}] {$\neg$}
                child { node[oval,label={[label distance=-2mm]80:$f_7$}] {$c$} }
              }
            }
          };
        \end{tikzpicture}
      \end{center}      
    \end{column}
    \begin{column}{0.2\textwidth}
    \end{column}
    \begin{column}{0.4\textwidth}
      \lstinputlisting[style=customConf]{../conf/example-6.22.frm}
    \end{column}
  \end{columns}
\end{frame}

\begin{frame}
  \frametitle{\acf{TS}}
  \begin{block}{Definizione}
    $TS = $\alert{$(S, Act, \longrightarrow, I, AP, L)$}
    \begin{itemize}
    \item $S$ insieme di \alert{stati}
    \item $Act$ insieme di \alert{azioni} (singola azione per \acs{CTL})
    \item $\longrightarrow\ $\alert{$\subseteq S\times Act\times S $}
    \item $I\subseteq S$ insieme di stati \alert{iniziali}
    \item $AP$ insieme di \alert{proposizioni atomiche}
    \item $L:$\alert{$S\rightarrow 2^{AP}$} 
    \end{itemize}
  \end{block}
  \begin{center}
    \begin{tikzpicture}
      \coordinate(init) at (-1,1);
      \node(s0) at (0,1) [oval, label={[label distance=-2mm]110:$s_0$}] {$a$};
      \node(s1) at (3,1) [oval, label={[label distance=-2mm]80:$s_1$}] {$a,b$};
      \node(s2) at (1.5,-1) [oval, label={[label distance=-2mm]190:$s_2$}] {$b$};
      \node(s3) at (5.5,1) [oval, label={[label distance=-2mm]110:$s_3$}] {$a$};
      \draw[arrow] (init) to (s0);
      \draw[arrow] (s0) to[bend left] (s1);
      \draw[arrow] (s1) to[bend left] (s0);
      \draw[arrow] (s0) to (s2);
      \draw[arrow] (s2) to (s1);
      \draw[arrow] (s1) to (s3);
      \draw[arrow] (s3) to[loop right] (s3);
    \end{tikzpicture}
  \end{center}
\end{frame}

\begin{frame}{fragile}
  \frametitle{Implementazione \ac{TS}}
  \begin{center}
    \begin{tikzpicture}
      \coordinate(init) at (-1,1);
      \node(s0) at (0,1) [oval, label={[label distance=-2mm]110:$s_0$}] {$a$};
      \node(s1) at (3,1) [oval, label={[label distance=-2mm]80:$s_1$}] {$a,b$};
      \node(s2) at (1.5,-1) [oval, label={[label distance=-2mm]190:$s_2$}] {$b$};
      \node(s3) at (5.5,1) [oval, label={[label distance=-2mm]110:$s_3$}] {$a$};
      \draw[arrow] (init) to (s0);
      \draw[arrow] (s0) to[bend left] (s1);
      \draw[arrow] (s1) to[bend left] (s0);
      \draw[arrow] (s0) to (s2);
      \draw[arrow] (s2) to (s1);
      \draw[arrow] (s1) to (s3);
      \draw[arrow] (s3) to[loop right] (s3);
    \end{tikzpicture}
  \end{center}
  \lstinputlisting[style=customConf]{../conf/book6.4.ts}
\end{frame}

\begin{frame}
  \frametitle{\acs{CTL} Model Checking}
  \begin{itemize}
  \item \alert{Verificare} se $TS\models\Phi$
    \begin{itemize}
    \item \alert{calcolare} ricorsivamente $Sat(\Phi)$\\[-7mm]
      \begin{eqnarray*}
        Sat(true)&=&S\\
        Sat(a)&=&\left\{s\in S| a\in L(s)\right\}\qquad,\forall a\in
        AP\\
        Sat(\Phi\land\Psi)&=&Sat(\Phi)\cap Sat(\Psi)\\
        Sat(\neg\Phi)&=&S\setminus Sat(\Phi)\\
        Sat(\exists\ctlNext\Phi)&=&\left\{s\in S|Post(s)\cap Sat(\Phi)\neq\varnothing\right\}\\
        Sat(\exists(\Phi\ctlUntil\Psi))&=&\text{il pi\`u piccolo }T\subseteq
        S\text{ t.c.}\\
        &&\quad Sat(\Psi)\cup\left\{s\in Sat(\Phi)|Post(s)\cap
          T\neq\varnothing\right\}\subseteq T\\
        Sat(\exists\ctlAlways\Phi)&=&\text{il pi\`u grande }T\subseteq
        S\text{ t.c.}\\
        &&\quad T\subseteq\left\{s\in Sat(\Phi)|Post(s)\cap
          T\neq\varnothing\right\}
      \end{eqnarray*}\\[-4mm]
      \begin{itemize}
      \item formule non \acs{ENF} vengono \alert{convertite}
      \end{itemize}
    \item $TS\models\Phi \Leftrightarrow I\subseteq Sat(\Phi)$
    \end{itemize}
  \end{itemize}
\end{frame}

\begin{frame}
  \frametitle{Complessit\`a \ac{CTL} Model Checking}
  \begin{block}{Algoritmi di base $Sat(\cdot)$, formula \ac{ENF}}
    $$
    \bigO((N+K)\cdot|\Phi|)
    $$
    \begin{itemize}
    \item $N$ numero di \alert{stati} del \ac{TS}
    \item $K$ numero di \alert{transizioni} del \ac{TS}
    \item $|\Phi|$ \alert{lunghezza} della formula $\Phi$
    \end{itemize}
  \end{block}
  \begin{block}{Complessit\`a $TS\models\Phi$}
    \begin{itemize}
    \item La trasformazione da formula \ac{CTL} generica a \ac{CTL}
      \ac{ENF} sarebbe \alert{esponenziale}, per\`o:
      \begin{itemize}
      \item Esistono algoritmi \alert{lineari} analoghi a quelli per formule \ac{ENF}
      \item Implementata conversione/calcolo \alert{lineare}
        di $Sat(\cdot)$ per formule generiche
      \end{itemize}
    \item La complessit\`a totale rimane \alert{invariata}
    \end{itemize}
  \end{block}
\end{frame}

\begin{frame}
  \frametitle{Formule non \ac{ENF}}
  \begin{itemize}
  \item Alberi di \alert{conversione} generici
    \begin{itemize}
    \item Foglie speciali per agganciarci alla formula originale
    \end{itemize}
  \end{itemize}
  \begin{center}
    \begin{tikzpicture}
      \node(f0) at (0,2) [oval, label={[label distance=-1mm]110:$f_0$}] {$\exists\ctlEventually$};
      \node(f1) at (0,0) [oval, label={[label distance=-1mm]110:$f_1$}] {$\Phi_o$};

      \node(ff0) at (4,2) [oval, label={[label distance=-1mm]110:$f_0$}] {$\exists\ctlUntil$};
      \node(ff1) at (3,0) [oval, label={[label distance=-1mm]110:$f_1$}] {$true$};
      \node(ff2) at (5,0) [oval, label={[label distance=-1mm]80:$f_2$}] {$\Phi_c$};
      \draw[bigArrow] (1,1) to (2,1);
      \draw[arrow] (f0) to (f1);
      \draw[arrow] (ff0) to (ff1);
      \draw[arrow] (ff0) to (ff2);
    \end{tikzpicture}
  \end{center}
  \begin{enumerate}
  \item \alert{Calcolo} di $Sat(\Phi_o)$
  \item \alert{Salva} risultato in nodo speciale $\Phi_c$
    \begin{itemize}
    \item Evita esplosione della formula $\forall(\Phi\ctlUntil\Psi)$
    \end{itemize}
  \item \alert{Calcolo} di $Sat(\exists\ctlUntil)$ su albero di conversione
  \end{enumerate}
\end{frame}

\begin{frame}
  \frametitle{$\forall\ctlUntil$ explosion}
  \begin{itemize}
  \item Formula da convertire:
  \end{itemize}
  \begin{center}
    \begin{tikzpicture}
      \node(f0) at (4,2) [oval, label={[label distance=-1mm]110:$f_0$}] {$\forall\ctlUntil$};
      \node(f1) at (3,0) [oval, label={[label distance=-1mm]110:$f_1$}] {$\Phi_o$};
      \node(f2) at (5,0) [oval, label={[label distance=-1mm]80:$f_2$}] {$\Psi_o$};
      \draw[arrow] (f0) to (f1);
      \draw[arrow] (f0) to (f2);
    \end{tikzpicture}
  \end{center}
\end{frame}

\begin{frame}
  \begin{center}
    \begin{tikzpicture}[scale=1.3]
      \node(f0) at (2,5) [oval, label={[label distance=-1mm]110:$f_0$}] {$\land$};
      \node(f1) at (1,4) [oval, label={[label distance=-1mm]110:$f_1$}] {$\neg$};
      \node(f2) at (3,4) [oval, label={[label distance=-1mm]80:$f_2$}] {$\neg$};
      \node(f3) at (1,3) [oval, label={[label distance=-1mm]110:$f_3$}] {$\exists\ctlUntil$};
      \node(f4) at (3,3) [oval, label={[label distance=-1mm]80:$f_4$}] {$\exists\ctlAlways$};
      \node(f5) at (0,2) [oval, label={[label distance=-1mm]110:$f_5$}] {$\neg$};
      \node(f6) at (2,2) [oval, label={[label distance=-1mm]80:$f_6$}] {$\land$};
      \node(f7) at (3,2) [oval, label={[label distance=-1mm]80:$f_7$}] {$\neg$};
      \node(f8) at (0,0) [oval, label={[label distance=-1mm]110:$f_8$}] {$\Psi_c'$};
      \node(f9) at (1,1) [oval, label={[label distance=-1mm]110:$f_9$}] {$\neg$};
      \node(f10) at (2,1) [oval, label={[label distance=-1mm]80:$f_{10}$}] {$\neg$};
      \node(f11) at (1,0) [oval, label={[label distance=-1mm]110:$f_{11}$}] {$\Phi_c$};
      \node(f12) at (2,0) [oval, label={[label distance=-1mm]80:$f_{12}$}] {$\Psi_c''$};
      \node(f13) at (3,0) [oval, label={[label distance=-1mm]80:$f_{13}$}] {$\Psi_c'''$};      
      
      \draw[arrow] (f0) to (f1);
      \draw[arrow] (f0) to (f2);
      \draw[arrow] (f1) to (f3);
      \draw[arrow] (f2) to (f4);
      \draw[arrow] (f3) to (f5);
      \draw[arrow] (f3) to (f6);
      \draw[arrow] (f4) to (f7);
      \draw[arrow] (f5) to (f8);
      \draw[arrow] (f6) to (f9);
      \draw[arrow] (f6) to (f10);
      \draw[arrow] (f7) to (f13);
      \draw[arrow] (f9) to (f11);
      \draw[arrow] (f10) to (f12);
    \end{tikzpicture}
  \end{center}
  \begin{itemize}
  \item $Sat(\Psi)$ da calcolare \alert{tre} volte
  \end{itemize}
\end{frame}

\begin{frame}
  \begin{center}
    \begin{tikzpicture}[scale=1.3]
      \node(f0) at (2,5) [oval, label={[label distance=-1mm]110:$f_0$}] {$\land$};
      \node(f1) at (1,4) [oval, label={[label distance=-1mm]110:$f_1$}] {$\neg$};
      \node(f2) at (3,4) [oval, label={[label distance=-1mm]80:$f_2$}] {$\neg$};
      \node(f3) at (1,3) [oval, label={[label distance=-1mm]110:$f_3$}] {$\exists\ctlUntil$};
      \node(f4) at (3,3) [oval, label={[label distance=-1mm]80:$f_4$}] {$\exists\ctlAlways$};
      \node(f5) at (0,2) [oval, label={[label distance=-1mm]110:$f_5$}] {$\neg$};
      \node(f6) at (2,2) [oval, label={[label distance=-1mm]80:$f_6$}] {$\land$};
      \node(f7) at (3,2) [oval, label={[label distance=-1mm]80:$f_7$}] {$\neg$};
      \node(f8) at (1,1) [oval, label={[label distance=-1mm]110:$f_8$}] {$\neg$};
      \node(f9) at (2,1) [oval, label={[label distance=-1mm]80:$f_9$}] {$\neg$};
      \node(f10) at (1,0) [oval, label={[label distance=-1mm]110:$f_{10}$}] {$\Phi_c$};
      \node(f11) at (3,0) [oval, label={[label distance=-1mm]80:$f_{11}$}] {$\Psi_c$};      
      
      \draw[arrow] (f0) to (f1);
      \draw[arrow] (f0) to (f2);
      \draw[arrow] (f1) to (f3);
      \draw[arrow] (f2) to (f4);
      \draw[arrow] (f3) to (f5);
      \draw[arrow] (f3) to (f6);
      \draw[arrow] (f4) to (f7);
      \draw[arrow] plot[smooth] coordinates{(f5.south) (0.5,-0.5) (f11.west)};
      \draw[arrow] (f6) to (f8);
      \draw[arrow] (f6) to (f9);
      \draw[arrow] (f7) to (f11);
      \draw[arrow] (f8) to (f10);
      \draw[arrow] (f9) to (f11);
    \end{tikzpicture}
  \end{center}
  \rule{0mm}{0mm}\\[-9mm]
  \begin{itemize}
  \item $Sat(\Psi)$ calcolata \alert{una} volta
  \end{itemize}
\end{frame}

\begin{frame}
  \begin{center}
	\textbf{\calligra\Huge The End.}\\
  \includegraphics[width=5cm]{img/ornament.eps}\\[1cm]
	{\huge\calligra Questions? Thank you!}
  \end{center}
\end{frame}

\appendix

\begin{frame}[fragile]
  \frametitle{Implementazione costruttore}
  %__init__
  \pfilePart{10}{39}{../ctlChecker.py}
\end{frame}

\begin{frame}[fragile]
  \frametitle{Implementazione $Sat(\cdot)$}
  %sat
  \pfilePart{200}{206}{../ctlChecker.py}  
  %_sat
  \pfilePart{191}{198}{../ctlChecker.py}  
\end{frame}

\begin{frame}[fragile]
  \frametitle{Implementazione $Sat(\cdot)$}
  %_satTrue
  \pfilePart{41}{46}{../ctlChecker.py}  
  %_satAp
  \pfilePart{48}{58}{../ctlChecker.py}  
  %_satAnd
  \pfilePart{60}{68}{../ctlChecker.py}  
\end{frame}

\begin{frame}[fragile]
  \frametitle{Implementazione $Sat(\cdot)$}
  %_satNot
  \pfilePart{70}{75}{../ctlChecker.py}  
  %_satExNext
  \pfilePart{77}{89}{../ctlChecker.py}  
\end{frame}

\begin{frame}[fragile]
  \frametitle{Implementazione $Sat(\cdot)$}
  %_satExUntil
  \pfilePart{91}{112}{../ctlChecker.py}  
\end{frame}

\begin{frame}[fragile]
  \frametitle{Implementazione $Sat(\cdot)$}
  %_satExAlways
  \pfilePart{115}{139}{../ctlChecker.py}  
\end{frame}

\begin{frame}[fragile]
  \frametitle{Implementazione $Sat(\cdot)$ formule non \acs{ENF}}
  %_satConversionOneSon
  \pfilePart{141}{152}{../ctlChecker.py}
  \rule{0mm}{0mm}\\[-5mm]
  %_satConversionTwoSons
  \pfilePart{154}{167}{../ctlChecker.py}  
\end{frame}

\begin{frame}[fragile]
  \frametitle{Implementazione $Sat(\cdot)$ formule non \acs{ENF}}
  %_satConversionTwoSonsOrdered
  \pfilePart{169}{182}{../ctlChecker.py}  
\end{frame}

\begin{frame}[fragile]
  \frametitle{Implementazione $TS\models\Phi$}
  %check
  \pfilePart{208}{216}{../ctlChecker.py}  
\end{frame}

\begin{frame}
  \frametitle{Complessit\`a rispetto a \acf{LTL}}
  \begin{itemize}
  \item[1.] \ac{HP} \`e \alert{NP-Completo}
    \begin{itemize}
    \item Trovare un percorso in un certo grafo $G$ che passa
      esattamente una volta per ogni nodo
    \end{itemize}
  \item[2a.] \`e possibile descrivere $TS_G$ e una formula
    \alert{\acs{LTL}} $\Phi_{LTL}$ di
    lunghezza \alert{polinomiale} nel numero di stati del grafo del problema
    \begin{itemize}
    \item  tali che
      $TS_G\not\models\Phi_{LTL}\quad\Leftrightarrow\quad G$
      contiene un percorso hamiltoniano
    \end{itemize}
  \item[2b.] il model checking di \ac{LTL} ha complessit\`a \alert{esponenziale} in $|\Phi|$
  \item[3a.] \`e possibile descrivere un $TS_G$ e una formula
    \alert{\ac{CTL}} $\Phi_{CTL}$ di lunghezza \alert{esponenziale} nel numero di
    stati del grafo
    \begin{itemize}
    \item  tali che
      $TS_G\not\models\neg\Phi_{CTL}\quad\Leftrightarrow\quad G$
      contiene un percorso hamiltoniano
    \end{itemize}
  \item[3b.] il model checking di \ac{CTL} ha complessit\`a \alert{lineare} in $|\Phi|$
  \item[4.] se $P\neq NP$ \alert{non} esiste una formula $\Phi_{CTL}$ equivalente
    a $\Phi_{LTL}$ e \alert{non} esponenzialmente pi\`u lunga.
  \end{itemize}
\end{frame}

\begin{frame}{fragile}
  \frametitle{Implementazione $TS_G$, $\Phi_{LTL}$, $\Phi_{CTL}$}
  \begin{center}
    \begin{tikzpicture}
      \clip (-1.7,-1.35) rectangle (5.7,1.75);

      \coordinate(i1) at (-1,1.8);
      \coordinate(i2) at (1.9,1);
      \coordinate(i3) at (2.1,1);
      \coordinate(i4) at (5,1.8);
      \node(s1) at (-1,1) [oval, label={[label distance=-1.8mm]110:$\upsilon_1$}] {$v_1$};
      \node(s2) at (1,1) [oval, label={[label distance=-1.8mm]110:$\upsilon_2$}] {$v_2$};
      \node(sb) at (2,-0.5) [oval, label={[label distance=-1.8mm]190:$\beta$}] {$b$};
      \node(s3) at (3,1) [oval, label={[label distance=-1.8mm]80:$\upsilon_3$}] {$v_3$};
      \node(s4) at (5,1) [oval, label={[label distance=-1.8mm]80:$\upsilon_4$}] {$v_4$};
      \draw[arrow] (i1) to (s1);
      \draw[arrow] (i2) to (s2);
      \draw[arrow] (i3) to (s3);
      \draw[arrow] (i4) to (s4);
      \draw[arrow] (s1) to[bend right] (sb);
      \draw[arrow] (s1) to (s2);
      \draw[arrow] (s1) to[bend left] (s3);
      \draw[arrow] (s2) to (sb);
      \draw[arrow] (s2) to[bend left] (s4);
      \draw[arrow] (s3) to (sb);
      \draw[arrow] (s3) to (s4);
      \draw[arrow] (s3) to[bend left] (s1);
      \draw[arrow] (s4) to[bend left] (sb);
      \draw[arrow] (s4) to[bend left] (s1);
      \draw[arrow] (sb) to[loop below] (sb);
    \end{tikzpicture}
  \end{center}
    \begin{eqnarray*}
      \Phi_{LTL}&=&\neg\landatory_{\upsilon\in
        V}(\ctlEventually\upsilon\land\ctlAlways(\upsilon\rightarrow\ctlNext\ctlAlways\neg\upsilon))\\
      \Phi_{CTL}&=&\lorratory_{(i_1,\dots,i_n)}\Psi(\upsilon_{i_1},\dots,\upsilon_{i_n})\quad\alert{\in\bigO(n!)}
    \end{eqnarray*}
    dove $n$ \`e il numero di stati, e $\Psi(\upsilon_i)=\upsilon_i;\quad\Psi(\upsilon_{i_1},\upsilon_{i_2},\dots,\upsilon_{i_n})=\upsilon_{i_1}\land\exists\ctlNext\Psi(\upsilon_{i_2},\dots,\upsilon_{i_n})$.
\end{frame}

\end{document}